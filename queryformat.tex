\documentclass{article}
\usepackage[utf8]{inputenc}

\usepackage{listings}
\usepackage{xcolor}

%New colors defined below
\definecolor{codegreen}{rgb}{0,0.6,0}
\definecolor{codegray}{rgb}{0.5,0.5,0.5}
\definecolor{codepurple}{rgb}{0.58,0,0.82}
\definecolor{backcolour}{rgb}{0.95,0.95,0.92}

%Code listing style named "mystyle"
\lstdefinestyle{mystyle}{
  backgroundcolor=\color{backcolour},   commentstyle=\color{codegreen},
  keywordstyle=\color{magenta},
  numberstyle=\tiny\color{codegray},
  stringstyle=\color{codepurple},
  basicstyle=\ttfamily\footnotesize,
  breakatwhitespace=false,         
  breaklines=true,                 
  captionpos=b,                    
  keepspaces=true,                 
  numbers=left,                    
  numbersep=5pt,                  
  showspaces=false,                
  showstringspaces=false,
  showtabs=false,                  
  tabsize=2
}

%"mystyle" code listing set
\lstset{style=mystyle}



\begin{document}
\noindent
\textbf{Query Submission Format}\\
\\
\noindent
Submit the sample queries in an output file called "sampleQueries.json". Submit the challenge queries in an output file called "challengeQueries.json". Create the output files using the format specified in Listing 1, below. If a student fails to follow this format, they will lose \textbf{1 mark}. 
\\
%Python code highlighting
\begin{lstlisting}[language=Python, caption=Python example]
import json
#Submit your queries in the following format:
# 1. <query1> -> replace with the corresponding query string 
# 2. [1,2,4] -> replace with the corresponding postings list
# 3. change the filename "sampleQueries.json" for the appropriate situation
q = {"<query1>": [1,2,4],"<query2>": [2,3],"<query3>": [1,4]}
json.dump(q, open("sampleQueries.json","w",encoding="utf-8"),indent=3)



\end{lstlisting}



\end{document}

